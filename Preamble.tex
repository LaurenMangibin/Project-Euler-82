
\usepackage{amscd}
\usepackage{amsfonts}
\usepackage{amsmath}
\usepackage{amsrefs}
\usepackage{amssymb}
\usepackage{amsthm}
\usepackage{array}
\usepackage{arydshln}
\usepackage{bbm}
\usepackage{calc}
\usepackage[labelformat=empty, font=bf]{caption}
\usepackage{color}
\usepackage{comment}
\usepackage{enumerate}
\usepackage{epsfig}
\usepackage{epstopdf}
\usepackage{etoolbox}
\usepackage{eucal}
\usepackage{fix-cm}
\usepackage{graphicx}
\usepackage{hyperref}
\usepackage[utf8]{inputenc}
\usepackage{latexsym}
\usepackage[procnames]{listings}
\usepackage{lscape}
\usepackage{mathrsfs}
\usepackage{multicol}
\usepackage{multirow}
\usepackage{pgfplots}
\usepackage{rotating}
\usepackage{setspace}
\usepackage{soul}
\usepackage{stmaryrd}
\usepackage{textcomp}
\usepackage[absolute]{textpos}
	\setlength{\TPHorizModule}{1in}
	\setlength{\TPVertModule}{\TPHorizModule}
	\textblockorigin{.35in}{.35in} % start everything near the top-left corner
\usepackage{tikz}
\usepackage{totcount}
\usepackage{verbatim}
\usepackage{vwcol}
\usepackage[all]{xy}

\newcommand{\ord}{\mathrm{ord}}

\newcommand{\rowheisize}{.35}
\newcommand{\rowhei}{\rule{0pt}{\rowheisize in}}

\usetikzlibrary{matrix, shapes, positioning, calc, decorations.pathreplacing, shapes.geometric, arrows}
\newcolumntype{x}[1]{>{\centering\arraybackslash\hspace{0pt}}p{#1}}

\input amssym.def

%%%%%%%%%%%%%%%%%%%%%%%%%%%%%%%%%%%%%%%%%%%%%%%%%%%%%%%%%
\newcommand{\inv}{^{-1}}
\newcommand{\N}{\mathbb{N}}
\newcommand{\R}{\mathbb{R}}
\newcommand{\Q}{\mathbb{Q}}
\newcommand{\Z}{\mathbb{Z}}
\newcommand{\C}{\mathbb{C}}
\newcommand{\A}{\mathcal{A}}
\newcommand{\I}{\mathcal{I}}
\DeclareMathOperator\cis{cis}
\DeclareMathOperator\Arg{Arg}
\DeclareMathOperator{\proj}{proj}

\newcommand{\hwreport}{
	\begin{textblock}{3}(0,0)
		\begin{tabular}{|p{1.5in}p{1.5in}p{1.1in}|}
		\hline
		&&\\ [5pt]
		Completion: \mun{.6}& Correctness: \mun{.6}& Total: \mun{.6}\\ [10pt]
		\multicolumn{2}{|l}{List of problems I couldn't do:} & \\ [10pt]
		\hline
		\end{tabular}
	\end{textblock}}

\def\interior{\mathop{\rm Int}\nolimits}
\def\num{\mathop{\#}\limits}
\def\bd{\mathop{\rm Bd}\nolimits}

\newcommand{\repindex}[4]{\left(\begin{tabular}{c|c} $#1$ & $#2$\\ $#3$ & $#4$\end{tabular}\right)}

\newcounter{Total}
\newcounter{Storage}

\newcommand{\mun}[1]{\makebox[#1in]{\hrulefill}}
\newcommand{\spac}{\vskip1.3in}
\newcounter{abscount}
\setcounter{abscount}{0}
\newcommand{\abs}[1]{\ifnumless{#1}{0}{\setcounter{abscount}{-#1}}
{\setcounter{abscount}{#1}}\arabic{abscount}}

%%%%%%%%%%%%%%%%%%%%%%%%%%%%%%%%%%%%%%%%%%%%%%%%%%%%%%%%%%
\newcommand{\drawpoly}[2]{
\begin{tikzpicture}[baseline=0,scale=.7]
\drawaxis

\foreach \i in {1,...,#1}
{\draw[ultra thick] (360/#1*\i-360/#1:\polygonsize) -- (360/#1*\i:\polygonsize);
}

\draw[ultra thick] (360/#1*#1:\polygonsize)-- (0:\polygonsize);

\ifnumequal{#2}{1}
{
\foreach \i in {1,...,#1}
{
\draw[thick,fill=white] (360/#1*\i-360/#1:\polygonsize) circle[radius=\circlesize];
\draw (360/#1*\i-360/#1:\polygonsize) node{\i};
}}{}

\end{tikzpicture}}
%%%%%%%%%%%%%%%%%%%%%%%%%%%%%%%%%%%%%%%%%%%%%%%%%%%%%%%%%%

\newcommand{\collen}{2in}
\newcommand{\polygonsize}{1.75 cm}
\newcommand{\circlesize}{.3}
\newcommand{\vertexsize}{.1in}

%%%%%%%%%%%%%%%%%%%%%%%%%%%%%%%%%%%%%%%%%%%%%%%%%%%%%%%%%%
% ASSESSMENT WITH MULTIPLE VERSIONS AND ANSWER KEYS

\newcommand{\Content}{ }
\newcommand{\ver}[2]{#1}
\newcommand{\ans}[2]{~\framebox[#1in]{\rule{0pt}{#2in}}}		% answer box
\newcommand{\answerswitch}[1]{{\color{white}{#1}}}			% answer key (with no box)
\newcommand{\vanswerswitch}[2]{{\color{white}{#1}} {\color{white}{#2}}}
\newcommand{\granswerswitch}[1]{\phantom{#1}}				% graph answer key
\newcommand{\vgranswerswitch}[2]{\phantom{#1} \phantom{#2}}
\newcommand{\aans}[3]{\framebox{\begin{minipage}[b][#2in][c]{#1in}{\begin{center}\answerswitch{#3}\end{center}}\end{minipage}}}	% answer box with key
\newcommand{\shaans}[3]{\framebox{\begin{minipage}[b][#2in][c]{#1in}{\begin{center}#3\end{center}}\end{minipage}}}
\newcommand{\asch}[1]{\aans{1}{1}{\resizebox{1in}{!}{#1}}}	% used for exam reviews?  Possibly not current.
\newcommand{\vans}[4]{\framebox{\begin{minipage}[b][#2in][c]{#1in}{\begin{center}\vanswerswitch{\ver{#3}{#4}}{\ver{\scriptsize #4}{\scriptsize #3}}\end{center}}\end{minipage}}}	% answer box with key for multiple versions
\newcommand{\vanswer}[2]{ 							% answer key (with no box) for multiple versions
	\vanswerswitch
		{\ver{#1}{#2}}
		{\ver{\scriptsize #2}{\scriptsize #1}}
}
\newcommand{\vgrans}[2]{ 							% graph answer key for multiple versions
	\vgranswerswitch
		{\ver{#1}{#2}}
		{\ver{#2}{#1}}
}

\newcommand{\Buildit}{
	\setcounter{PointsPossible}{0}\setcounter{CalcPointsPossible}{0}
	\Content
	\newpage\setcounter{page}{1} 
	\renewcommand{\answerswitch}[1]{{\color{red}##1}} 
	\renewcommand{\vanswerswitch}[2]{{\color{red}{##1}} {\color{blue}{##2}}}
	\renewcommand{\granswerswitch}[1]{{\color{red}##1}}
	\renewcommand{\vgranswerswitch}[2]{{\color{blue!40}{##2}}{\color{red}{##1}}}
	\setcounter{PointsPossible}{0}\setcounter{CalcPointsPossible}{0}
	\Content
	\renewcommand{\answerswitch}[1]{{\color{white}##1}} 
	\renewcommand{\vanswerswitch}[2]{{\color{white}{##1}} {\color{white}{##2}}}
	\renewcommand{\granswerswitch}[1]{\phantom{##1}}
	\renewcommand{\vgranswerswitch}[2]{\phantom{##1} \phantom{##2}}
}

\newcommand{\Buildem}{
	\Buildit
	\renewcommand{\ver}[2]{##2}
	\newpage\setcounter{page}{1}
	\Buildit
}
%%%%%%%%%%%%%%%%%%%%%%%%%%%%%%%%%%%%%%%%%%%%%%%%%%%%%%%%%%
%SEPARATE ANSWER KEY FOR CRYPTOLOGY PACKET
\newcommand{\QA}[1]{}
\newcounter{RubricPageCount}
\setcounter{RubricPageCount}{0}

\newcommand{\CPSrubric}{
	\newpage
	\setcounter{RubricPageCount}{\value{page}}
	\subsection{Section \thesection{} Grading Rubric}	
	\thispagestyle{empty}
	\pagestyle{empty}
	\begin{itemize}
	\item \textbf{Completion} (50\%) \hfill \mun{.75}/50 \\
	There are \theenumi{} problems in this section, so they are worth 	
	{\color{red} \pgfmathtruncatemacro{\worth}{50/\theenumi}\worth{} 		points each}.
	\item \textbf{Correctness} (30\%) \hfill \mun{.75}/30
	\begin{multicols}{2}
	\begin{footnotesize}
	\renewcommand{\QA}[1]{\begin{quote}{\color{red}##1}\end{quote}}
	\setcounter{enumi}{0}
	\Content
	\renewcommand{\QA}[1]{}
	\end{footnotesize}
	\end{multicols}
	\item \textbf{Thoroughness} (10\%) \hfill \mun{.75}/10 \\
	This should reflect the extent to which you justified your answers 	(i.e., ``showed your work").	
	\item \textbf{Presentation} (10\%) \hfill \mun{.75}/10 \\
	This number will be inversely proportional to my level of 				frustration upon trying to read your work.\\
	\hrule \vspace{.25in}
	\hfill \textbf{Submission grade}: \mun{.675}/100
	\end{itemize}
	\addtocounter{page}{-1} % FIX (account for multipage rubric)
	\newpage
	\setcounter{page}{\value{RubricPageCount}}
	\pagestyle{headings}
	}

\newcommand{\LogicRubric}{
	\newpage
	\setcounter{RubricPageCount}{\value{page}}
	\subsection{Section \thesection{} Grading Rubric}	
	\thispagestyle{empty}
	\pagestyle{empty}
	\begin{itemize}
	\item \textbf{Correctness} (80\%) \hfill \mun{.75}/80 \\
	There are \theenumi{} problems in this section, so they are 
	worth 	
	{\color{red} \pgfmathtruncatemacro{\worth}{80/\theenumi}\worth{} 	points each}.
	\begin{multicols}{2}
	\begin{footnotesize}
	\renewcommand{\QA}[1]{\begin{quote}{\color{red}##1}\end{quote}}
	\setcounter{enumi}{0}
	\Content
	\renewcommand{\QA}[1]{}
	\end{footnotesize}
	\end{multicols}
	\item \textbf{Thoroughness} (10\%) \hfill \mun{.75}/10 \\
	This should reflect the extent to which you justified your answers 	(i.e., ``showed your work").	
	\item \textbf{Presentation} (10\%) \hfill \mun{.75}/10 \\
	This number will be inversely proportional to my level of 				frustration upon trying to read your work.\\
	\hrule \vspace{.25in}
	\hfill \textbf{Submission grade}: \mun{.675}/100
	\end{itemize}
	%\addtocounter{page}{-1} % FIX (account for multipage rubric)
	\newpage
	\setcounter{page}{\value{RubricPageCount}}
	\pagestyle{headings}
	}

\newcommand{\LogicRubricNoMulticols}{
	\newpage
	\setcounter{RubricPageCount}{\value{page}}
	\subsection{Section \thesection{} Grading Rubric}	
	\thispagestyle{empty}
	\pagestyle{empty}
	\begin{itemize}
	\item \textbf{Correctness} (80\%) \hfill \mun{.75}/80 \\
	There are \theenumi{} problems in this section, so they are 
	worth 	
	{\color{red} \pgfmathtruncatemacro{\worth}{80/\theenumi}\worth{} 	points each}.
	\begin{footnotesize}
	\renewcommand{\QA}[1]{\begin{quote}{\color{red}##1}\end{quote}}
	\setcounter{enumi}{0}
	\Content
	\renewcommand{\QA}[1]{}
	\end{footnotesize}
	\item \textbf{Thoroughness} (10\%) \hfill \mun{.75}/10 \\
	This should reflect the extent to which you justified your answers 	(i.e., ``showed your work").	
	\item \textbf{Presentation} (10\%) \hfill \mun{.75}/10 \\
	This number will be inversely proportional to my level of 				frustration upon trying to read your work.\\
	\hrule \vspace{.25in}
	\hfill \textbf{Submission grade}: \mun{.675}/100
	\end{itemize}
	%\addtocounter{page}{-1} % FIX (account for multipage rubric)
	\newpage
	\setcounter{page}{\value{RubricPageCount}}
	\pagestyle{headings}
	}
%%%%%%%%%%%%%%%%%%%%%%%%%%%%%%%%%%%%%%%%%%%%%%%%%%%%%%%%%%

\newcommand{\colorchange}{white}

%%%%%%%%%%%%%%%%%%%%%%%%%%%%%%%%%%%%%%%%%%%%%%%%%%%%%%%%%%

\newcommand{\axisgridnumbers}[2]
{
\begin{tikzpicture}[baseline=0]
\draw (-#1,-#1) grid (#1,#1);

\draw[<->,ultra thick] (-#1-.3,0) -- (#1+.3,0);
\draw[<->,ultra thick] (0,-#1-.3) -- (0,#1+.3);

\foreach \i in {1,...,#1}
{\draw[ultra thick] (-.2,\i)  -- (.2,\i) node[right, fill=white]{\i};}
\foreach \i in {1,...,#1}
{\draw[ultra thick] (-.2,-\i)  -- (.2,-\i) node[right, fill=white]{-\i};}
\foreach \i in {1,...,#1}
{\draw[ultra thick] (\i,.2)  -- (\i,-.2) node[below, fill=white]{\i};}
\foreach \i in {1,...,#1}
{\draw[ultra thick] (-\i,.2)  -- (-\i,-.2) node[below, fill=white]{-\i};}
\granswerswitch{#2}

\end{tikzpicture}
}
%%%%%%%%%%%%%%%%%%%%%%%%%%%%%%%%%%%%%%%%%%%%%%%%%%%%%%%%%%
\newcommand{\filledaxisgridnumbers}[2]
{
\begin{tikzpicture}[baseline=0]
\draw (-#1,-#1) grid (#1,#1);

\draw[<->,ultra thick] (-#1-.3,0) -- (#1+.3,0);
\draw[<->,ultra thick] (0,-#1-.3) -- (0,#1+.3);

\foreach \i in {1,...,#1}
{\draw[ultra thick] (-.2,\i)  -- (.2,\i) node[right, fill=white]{\i};}
\foreach \i in {1,...,#1}
{\draw[ultra thick] (-.2,-\i)  -- (.2,-\i) node[right, fill=white]{-\i};}
\foreach \i in {1,...,#1}
{\draw[ultra thick] (\i,.2)  -- (\i,-.2) node[below, fill=white]{\i};}
\foreach \i in {1,...,#1}
{\draw[ultra thick] (-\i,.2)  -- (-\i,-.2) node[below, fill=white]{-\i};}

#2
\end{tikzpicture}
}

%%%%%%%%%%%%%%%%%%%%%%%%%%%%%%%%%%%%%%%%%%%%%%%%%%%%%%%%%%
\newcommand{\RevQues}[3]{

{\fontsize{25}{30}\selectfont #1}\newpage

\begin{center}{\fontsize{#3}{#3}\selectfont #2}\end{center}

\vskip.1in\hrule\vskip.5in
}

%%%%%%%%%%%%%%%%%%%%%%%%%%%%%%%%%%%%%%%%%%%%%%%%%%%%%%%%%%
\newcounter{GraphLetter}\stepcounter{GraphLetter}
\newcommand{\PrintGraphLetter}
{{\fontsize{25}{30}\selectfont\centering\textbf{Graph \Alph{GraphLetter}}\newline }\stepcounter{GraphLetter}}

%%%%%%%%%%%%%%%%%%%%%%%%%%%%%%%%%%%%%%%%%%%%%%%%%%%%%%%%%%
\newcommand{\filledaxisgrid}[1]
{
\begin{tikzpicture}

\draw[step=.25] (-1.25,-1.25) grid (1.25,1.25);

\draw[<->, ultra thick] (-1.45,0) -- (1.45,0);
\draw[<->, ultra thick] (0,-1.45) -- (0,1.45);

\draw [ultra thick] (0,-.1)  -- (0,0.1);
\draw [ultra thick] (.25,-.1)  -- (.25,0.1);
\draw [ultra thick] (.5,-.1)  -- (.5,0.1);
\draw [ultra thick] (.75,-.1)  -- (.75,0.1);
\draw [ultra thick] (1,-.1)  -- (1,0.1);
\draw [ultra thick] (1.25,-.1)  -- (1.25,0.1);
\draw [ultra thick] (-.25,-.1) -- (-.25,0.1);
\draw [ultra thick] (-.5,-.1) -- (-.5,0.1);
\draw [ultra thick] (-.75,-.1) -- (-.75,0.1);
\draw [ultra thick] (-1,-.1)  -- (-1,0.1);
\draw [ultra thick] (-1.25,-.1) -- (-1.25,0.1);

\draw [ultra thick] (-.1,.25)  -- (0.1,.25);
\draw [ultra thick] (-.1,.5)  -- (0.1,.5);
\draw [ultra thick] (-.1,.75)  -- (0.1,.75);
\draw [ultra thick] (-.1,1)  -- (0.1,1);
\draw [ultra thick] (-.1,1.25)  -- (0.1,1.25);

\draw [ultra thick] (-.1,-.25)  -- (0.1,-.25);
\draw [ultra thick] (-.1,-.5)  -- (0.1,-.5);
\draw [ultra thick] (-.1,-.75) -- (0.1,-.75);
\draw [ultra thick] (-.1,-1) -- (0.1,-1);
\draw [ultra thick] (-.1,-1.25) -- (0.1,-1.25);

#1

\end{tikzpicture}
}

%%%%%%%%%%%%%%%%%%%%%%%%%%%%%%%%%%%%%%%%%%%%%%%%%%%%%%%%%%

% COUNTER FOR POINTS POSSIBLE ON TESTS
\newtotcounter{PointsPossible}
\newcommand{\points}[2]{ 				%\points{<number of points per problem>}{<number of problems>}
	\pgfmathtruncatemacro{\howmany}{#1*#2}
	\addtocounter{PointsPossible}{\howmany}
	(#1 \ifnumcomp{#1}{=}{1}{point}{points}\ifnumcomp{#2}{=}{1}{)}{ each)}		
		%you don't have to type ``point" or ``points"
		%if # of problems is > 1, it will print ``point(s) each"
}

\newtotcounter{CalcPointsPossible}
\newcommand{\calcpoints}[2]{ 				%\points{<number of points per problem>}{<number of problems>}
	\pgfmathtruncatemacro{\howmany}{#1*#2}
	\addtocounter{CalcPointsPossible}{\howmany}
	(#1 \ifnumcomp{#1}{=}{1}{point}{points}\ifnumcomp{#2}{=}{1}{)}{ each)}		
		%you don't have to type ``point" or ``points"
		%if # of problems is > 1, it will print ``point(s) each"
}

% BC PRECALCULUS TEST INSTRUCTIONS/HEADER
\newcommand{\NonCalculatorSection}{
	\begin{center} \textbf{Non-calculator section}\\ [5pt]
	\fbox{\fbox{\parbox{\textwidth}{Answers provided without supporting work will not receive credit.  There are
	\total{PointsPossible} points possible on the non-calculator section and \total{CalcPointsPossible} points possible on the
	calculator section, excluding bonus problems.}}}\end{center}}

\newcommand{\CalculatorSection}{
	\begin{center} \textbf{Calculator section}\\ [5pt]
	\fbox{\fbox{\parbox{\textwidth}{Answers provided without supporting work will not receive credit.  There are 					\total{PointsPossible} points possible on the non-calculator section and \total{CalcPointsPossible} points possible on the 				calculator section, excluding bonus problems. \textbf{Round all approximations to the nearest thousandth}.}}}\end{center}}

\newcommand{\CalculatorTestInstructions}{\NonCalculatorSection}

\newcommand{\BCPCTestTitle}[1]{
	{\fontsize{15}{18}\selectfont\textbf{BC Precalculus}} \hfill \textbf{Name:} \mun{2} \\ [12pt]
	{\fontsize{15}{18}\selectfont\textbf{Exam \testnumber: \testname}} \hfill \textbf{Period:} \mun{2} \\
	\ifstrequal{#1}{}
		{\begin{center}\fbox{\fbox{\parbox{\textwidth}{You may not use a calculator.  Answers provided without supporting work will not receive credit.  There are \total{PointsPossible} points possible (excluding the bonus).}}}\end{center}} 
		{\CalculatorTestInstructions}
	\thispagestyle{empty}
}

\newcommand{\LogicTestTitle}{
	{\fontsize{15}{18}\selectfont\textbf{Logic}} \hfill
	 \textbf{Name:} \mun{2} \\ [12pt]
	{\fontsize{15}{18}\selectfont\textbf{Test \testnumber:
	 \testname}} \hfill \textbf{Period:} \mun{2} \\
	\begin{center}\fbox{\fbox{\parbox{\textwidth}{Make sure you
	 justify all claims you make and explain yourself thoroughly.
	 There are \total{PointsPossible} points possible (excluding the
	 bonus).}}}\end{center} 
	\thispagestyle{empty}
	}

\newcommand{\CPSExamTitle}[2]{
	\begin{flushleft}
	{\fontsize{15}{18}\selectfont\textbf{Computational Problem Solving}} \hfill \textbf{Name:} \mun{2} \\ [12pt]
	{\fontsize{15}{18}\selectfont\textbf{Exam \testnumber: \testname}} \\ \vspace{-16pt}
	\end{flushleft} 
	\begin{center}\fbox{\fbox{\parbox{\textwidth}{Answer all questions by bubbling on your answer sheet.  Write your name
	 on \textbf{both} the answer sheet and the exam itself. \textbf{No credit} will be given for work done on the pages with the
	 questions, but you may write on it.  There are \MCVer{#1}{#2} questions, each worth 5 points. You may not use a
	 calculator.}}}\end{center}
	\thispagestyle{empty}
}

\newcommand{\CPSGuidelines}[2]{
	\hrule
	\textbf{How to get credit for this #1:}
	#2
}

\newcommand{\CPSShowMe}{
	\begin{enumerate}
	\item Show me your finished work \textbf{during class} at an opportune time.  You may have to explain things to me or make
	 some edits that I suggest.
	\item Do \textbf{NOT} send me anything electronically or expect me to grade a hard copy of something in your absence.  The
	 only acceptable way to receive credit is to follow the procedure above.
	\end{enumerate}
}

\newcommand{\CPSProveATheorem}{
	\begin{enumerate}
	\item Prove the theorem(s) listed above.
	\item Typset your proof in \LaTeX{}.  Include your name, the project title, the statement of the theorem(s), and your proof(s)
	 in your document.
	\item Show me your document (digitally or as a hard copy) \textbf{during class} at an opportune time.  You will have to walk
	 me through your proof, explaining to me why it works.
	\item Do \textbf{NOT} send me anything electronically or expect me to grade a hard copy of something in your absence.  The
	 only acceptable way to receive credit is to follow the procedure above.
	\end{enumerate}
}

\newcommand{\CPSSolveThis}{
	\begin{enumerate}
	\item Solve the problem(s) listed above.
	\item Typset your solution in \LaTeX{}.  Include your name, the project title, the statement of the problem(s), your answer,
	 and a thorough explanation of your method or rationale in your document.
	\item Show me your document (digitally or as a hard copy) \textbf{during class} at an opportune time.  You will have to walk
	 me through your solution, explaining to me why it works.
	\item Do \textbf{NOT} send me anything electronically or expect me to grade a hard copy of something in your absence.  The
	 only acceptable way to receive credit is to follow the procedure above.
	\end{enumerate}
}

\newcommand{\CPSTypesetSomething}{
	\begin{enumerate}
	\item Typset your document in \LaTeX{}.  \textbf{You must use my preamble} with no additional packages or commands (unless
	 we have spoken about it).
	\item When you are finished, show me your .tex file, .pdf file, and your original paper document \textbf{during class} at an
	 opportune time.  I may require you to revise some of your work.
	\item Once you've received my verbal approval, send me one email with your .tex and .pdf files attached.
	\item Do \textbf{NOT} send me anything unbeckoned electronically or expect me to grade a hard copy of something in your
	 absence.  The only acceptable way to receive credit is to follow the procedure above.
	\end{enumerate}
}

\newcommand{\BeginCalculatorSection}{
	\renewcommand{\CalculatorTestInstructions}{\CalculatorSection}
	\BCPCTestTitle{calc}
	\renewcommand{\CalculatorTestInstructions}{\NonCalculatorSection}
}

%%%%%%%%%%%%%%%%%%%%%%%%%%%%%%%%%%%%%%%%%%%%%%%%%%%%%%%%%%

% MULTIPLE CHOICE
\newcommand{\mctwo}[2]{
	\begin{multicols}{2}
		\begin{enumerate}[A.]
			\item #1
			\item #2
		\end{enumerate}
	\end{multicols}}
\newcommand{\mcthree}[3]{
	\begin{multicols}{2}
		\begin{enumerate}[A.]
			\item #1
			\item #2
			\item #3
		\end{enumerate}
	\end{multicols}}
\newcommand{\mcfour}[4]{
	\begin{multicols}{2}
		\begin{enumerate}[A.]
			\item #1
			\item #2
			\item #3
			\item #4
		\end{enumerate}
	\end{multicols}}
\newcommand{\mc}[5]{
	\begin{multicols}{2}
		\begin{enumerate}[A.]
			\item #1
			\item #2
			\item #3
			\item #4
			\item #5
		\end{enumerate}
	\end{multicols}}

%%%%%%%%%%%%%%%%%%%%%%%%%%%%%%%%%%%%%%%%%%%%%%%%%%%%%%%%%%

% HIGHLIGHT PYTHON SYNTAX USING THE LISTINGS PACKAGE (CAN DO OTHER LANGUAGES TOO)
% settings to mimmic Idle: http://blog.miliauskas.lt/2008/09/python-syntax-highlighting-in-latex.html

% Custom colors
\definecolor{orange}{rgb}{1,0.5,0}
\definecolor{lightgreen}{rgb}{0,0.7,0}
\definecolor{gray}{gray}{0.5}
\definecolor{green}{rgb}{0,0.5,0}
\definecolor{lightgreen}{rgb}{0,0.7,0}
\definecolor{purple}{rgb}{0.5,0,0.5}

% Python style for highlighting
\newcommand\pythonstyle{
\lstset{
  backgroundcolor=\color{white},	% choose the background color; you must add \usepackage{color} or \usepackage{xcolor}
  basicstyle=\ttfamily\small,        		% the size of the fonts that are used for the code
%  breakatwhitespace=false,         	% sets if automatic breaks should only happen at whitespace
%  breaklines=true,                 		% sets automatic line breaking
%  captionpos=b,                    		% sets the caption-position to bottom
  commentstyle=\color{red},    		% comment style
%  deletekeywords={...},            	% if you want to delete keywords from the given language
%  escapeinside={\%*}{*)},          	% if you want to add LaTeX within your code
%  extendedchars=true,              	% lets you use non-ASCII characters; for 8-bits encodings only, does not work with UTF-8
  frame=tb,	                   			% adds a frame around the code, also could use single for box
%  keepspaces=true,                 		% keeps spaces in text, useful for keeping indentation of code (possibly needs 										% columns=flexible)
  keywordstyle=\color{orange},       	% keyword style
  keywords=[2]{sum,range},		% manually add keywords you want to render in purple 
  keywordstyle={[2]\ttfamily\color{purple}},
  language=Python,                 		% the language of the code
  morecomment=[s][\color{lightgreen}]{"""}{"""},
  numbers=left,                    		% where to put the line-numbers; possible values are (none, left, right)
  numbersep=5pt,                   		% how far the line-numbers are from the code
  numberstyle=\tiny\color{gray}, 	% the style that is used for the line-numbers
  otherkeywords={self},            		% if you want to add more keywords to the set
procnamekeys={def,class},
procnamestyle=\color{blue}\textbf,
%  rulecolor=\color{black},         		% if not set, the frame-color may be changed on line-breaks within not-black text (e.g. 								% comments (green here))
%  showspaces=false,                		% show spaces everywhere adding particular underscores; it overrides 'showstringspaces'
  showstringspaces=false,          		% underline spaces within strings only
%  showtabs=false,                  		% show tabs within strings adding particular underscores
%  stepnumber=2,                    		% the step between two line-numbers. If it's 1, each line will be numbered
  stringstyle=\color{lightgreen},     		% string literal style
  tabsize=3,	                   			% sets default tabsize to that number of spaces
%  title=\lstname                   		% show the filename of files included with \lstinputlisting; also try caption instead of title
}}

% Python environment
\lstnewenvironment{python}[1][]
{
\pythonstyle
\lstset{#1}
}
{}

% Python for external files
\newcommand\pythonexternal[2][]{{
\pythonstyle
\lstinputlisting[#1]{#2}}}

% Python for inline
\newcommand\pythoninline[1]{\unskip\pythonstyle\lstinline!#1!}


% HIGHLIGHT PYTHON SYNTAX USING THE LISTINGS PACKAGE

\newcommand\latexstyle{
\lstset{
    language=[LaTeX]TeX,
    breaklines=true,
    basicstyle=\tt\small,
    keywordstyle=\color{blue},
    identifierstyle=\color{black},
    commentstyle=\color{gray}
}}

% LaTeX environment
\lstnewenvironment{latexcode}[1][]
{
\latexstyle
\lstset{#1}
}
{}

% LaTeX for external files
\newcommand\latexexternal[2][]{{
\latexstyle
\lstinputlisting[#1]{#2}}}

% LaTeX for inline
\newcommand\latexinline[1]{\unskip\latexstyle\lstinline!#1!}


%%%%%%%%%%%%%%%%%%%%%%%%%%%%%%%%%%%%%%%%%%%%%%%%%%%%%%%%%%

% FOR THE FUNCTION PLOT
\gdef\aster{*}
\newcount\interpoints
\newcount\iDomains
\newcount\jDomains

\long\def\domainlist #1*{
    \def\wstuff{#1}
    \expandafter\ulword\wstuff*
}

\long\def\ulword#1 {
    \def\one{#1}
    \ifx\one\aster\let\go\relax
    \else\vtop{
        \global\advance\interpoints by1
        \expandafter\gdef\csname domains\romannumeral\interpoints\endcsname{#1}
        %\immediate\write\sendtoaux{\noexpand\csname domains\romannumeral\interpoints\noexpand\endcsname}
        \relax
    }
    \let\go\ulword
    \fi\go
}
%%%%%%%%%%%%%%%%%%%%%%%%%%%%%%%%%%%%%%%%%%%%%%%%%%%%%%%%%%%%%%%%%%%%%%%
\newcommand{\function}[3]
{
\global\interpoints=0
%You need a space before the * here
\domainlist #3 *
\global\iDomains=1
\global\jDomains=2

\pgfmathsetmacro{\numintervals}{\interpoints/2}
\foreach \k in {1,...,\numintervals}
{

\pgfmathsetmacro{\leftpoint}{\csname domains\romannumeral\iDomains\noexpand\endcsname}
\pgfmathsetmacro{\rightpoint}{\csname domains\romannumeral\jDomains\noexpand\endcsname}
\global\advance\iDomains by2
\global\advance\jDomains by2

\draw[#1,<->,samples=100,domain=\leftpoint:\rightpoint] plot(\x,{#2});
}
}
%%%%%%%%%%%%%%%%%%%%%%%%%%%%%%%%%%%%%%%%%%%%%%%%%%%%%%%%%%%%%%%%%%%%%%%
\newcommand{\nafunction}[3]
{
\interpoints=0
%You need a space before the * here
\domainlist #3 *
\global\iDomains=1
\global\jDomains=2

\pgfmathsetmacro{\numintervals}{\interpoints/2}
\foreach \k in {1,...,\numintervals}
{

\pgfmathsetmacro{\leftpoint}{\csname domains\romannumeral\iDomains\noexpand\endcsname}
\pgfmathsetmacro{\rightpoint}{\csname domains\romannumeral\jDomains\noexpand\endcsname}
\global\advance\iDomains by2
\global\advance\jDomains by2

\draw[#1,samples=100,domain=\leftpoint:\rightpoint] plot(\x,{#2});
}
}
%%%%%%%%%%%%%%%%%%%%%%%%%%%%%%%%%%%%%%%%%%%%%%%%%%%%%%%%%%%%%%%%%%%%%%%

% MULTIPLE CHOICE ASSESSMENTS
\newcommand{\StandScale}{.275}
\newcommand{\SmallScale}{.25}
\newcommand{\sg}{\vskip.2in}

\newcommand{\MCBreak}{
\vspace*{\fill}\columnbreak
}

\newcommand{\MCExamStart}[3]
{
#1
\begin{center}
\fbox{\fbox{\parbox{6in}{%\centering
Answer all questions by bubbling on your answer sheet.  Write your name on \textbf{both} the answer sheet and the exam itself. \textbf{No credit} will be given for work done on the pages with the questions, but you may write on it.  There are \MCVer{#2}{#3} questions, each worth 2 points. You may not use a calculator. Unless stated otherwise, all tick marks are at intervals of 1.}}}
\end{center}
}

\newcommand{\FinalExamStart}
{\begin{center}
\fbox{\fbox{\parbox{6in}{%\centering
Answer all questions by bubbling on your answer sheet.  Write your name on \textbf{both} the answer sheet and the exam itself. \textbf{No credit} will be given for work done on the pages with the questions, but you may write on it.  There are \MCVer{fifty}{50} questions, each worth 2 points. You may not use a calculator. Unless stated otherwise, all tick marks are at intervals of 1.}}}
\end{center}
}

\newcommand{\BCFinalExamStart}[1]
{
\bctitle{#1}{0}
\FinalExamStart
}

\newcommand{\MCStart}{
\newcount\MCAnsCount \MCAnsCount=0
\newcount\MCSecCount \MCSecCount=1
\newcount\MCProblemCount \MCProblemCount=1
}

\newcommand{\MCRestart}{
\MCSecCount=1
\MCProblemCount=1
}

\newcommand{\MCSection}[2]
{\hrule \vskip.2in Section \number\MCSecCount.\space #1\sg\hrule
\begin{multicols}{2}
\begin{enumerate}
\ifnum\MCSecCount=1
\else
    \setcounter{enumi}{\MCProblemCount}
    \addtocounter{enumi}{-1}
\fi
#2
%\setcounter{tempcounter}{\value{enumi}}
\end{enumerate}
\end{multicols}
\global\advance\MCSecCount by1
}

\newcommand{\MCEmptySection}[1]
{
\begin{multicols}{2}
\begin{enumerate}
\ifnum\MCSecCount=1
\else
    \setcounter{enumi}{\MCProblemCount}
    \addtocounter{enumi}{-1}
\fi
#1
%\setcounter{tempcounter}{\value{enumi}}
\end{enumerate}
\end{multicols}
\global\advance\MCSecCount by1
}


\newcommand{\MCAns}[6]{
\item #1
\begin{enumerate}
\ifnumequal{\MCVerNum}{1}{
    \item #2
    \item #3
    \item #4
    \item #5
    \ifnumequal{#6}{1}
    {\expandafter\gdef\csname MCAns\romannumeral\MCProblemCount\endcsname{A}}{}
    \ifnumequal{#6}{2}
    {\expandafter\gdef\csname MCAns\romannumeral\MCProblemCount\endcsname{B}}{}
    \ifnumequal{#6}{3}
    {\expandafter\gdef\csname MCAns\romannumeral\MCProblemCount\endcsname{C}}{}
    \ifnumequal{#6}{4}
    {\expandafter\gdef\csname MCAns\romannumeral\MCProblemCount\endcsname{D}}{}
}{
    \ifnum\MCAnsCount=0
        \item #2
        \item #3
        \item #4
        \item #5
        \ifnumequal{#6}{1}
        {\expandafter\gdef\csname MCAns\romannumeral\MCProblemCount\endcsname{A}}{}
        \ifnumequal{#6}{2}
        {\expandafter\gdef\csname MCAns\romannumeral\MCProblemCount\endcsname{B}}{}
        \ifnumequal{#6}{3}
        {\expandafter\gdef\csname MCAns\romannumeral\MCProblemCount\endcsname{C}}{}
        \ifnumequal{#6}{4}
        {\expandafter\gdef\csname MCAns\romannumeral\MCProblemCount\endcsname{D}}{}
    \else
    \fi

    \ifnum\MCAnsCount=1
        \item #5
        \item #2
        \item #3
        \item #4
        \ifnumequal{#6}{1}
        {\expandafter\gdef\csname MCAns\romannumeral\MCProblemCount\endcsname{B}}{}
        \ifnumequal{#6}{2}
        {\expandafter\gdef\csname MCAns\romannumeral\MCProblemCount\endcsname{C}}{}
        \ifnumequal{#6}{3}
        {\expandafter\gdef\csname MCAns\romannumeral\MCProblemCount\endcsname{D}}{}
        \ifnumequal{#6}{4}
        {\expandafter\gdef\csname MCAns\romannumeral\MCProblemCount\endcsname{A}}{}
    \else
    \fi

    \ifnum\MCAnsCount=2
        \item #3
        \item #2
        \item #5
        \item #4
        \ifnumequal{#6}{1}
        {\expandafter\gdef\csname MCAns\romannumeral\MCProblemCount\endcsname{B}}{}
        \ifnumequal{#6}{2}
        {\expandafter\gdef\csname MCAns\romannumeral\MCProblemCount\endcsname{A}}{}
        \ifnumequal{#6}{3}
        {\expandafter\gdef\csname MCAns\romannumeral\MCProblemCount\endcsname{D}}{}
        \ifnumequal{#6}{4}
        {\expandafter\gdef\csname MCAns\romannumeral\MCProblemCount\endcsname{C}}{}
    \else
    \fi

    \ifnum\MCAnsCount=3
        \item #4
        \item #5
        \item #2
        \item #3
        \ifnumequal{#6}{1}
        {\expandafter\gdef\csname MCAns\romannumeral\MCProblemCount\endcsname{C}}{}
        \ifnumequal{#6}{2}
        {\expandafter\gdef\csname MCAns\romannumeral\MCProblemCount\endcsname{D}}{}
        \ifnumequal{#6}{3}
        {\expandafter\gdef\csname MCAns\romannumeral\MCProblemCount\endcsname{A}}{}
        \ifnumequal{#6}{4}
        {\expandafter\gdef\csname MCAns\romannumeral\MCProblemCount\endcsname{B}}{}
    \else
    \fi

    \ifnum\MCAnsCount=3
        \MCAnsCount=0
    \else
        \global\advance\MCAnsCount by1
    \fi

}
\end{enumerate}
\global\advance\MCProblemCount by1
}


\newcommand{\MCKey}[1]{
	\newpage
	\thispagestyle{empty}
	\begin{tightcenter} {\Huge KEY #1}\end{tightcenter}
	\pgfmathsetmacro{\AnsKeyNum}{\MCProblemCount-1}
	\begin{multicols}{2}
	\begin{enumerate}
	\foreach \i in {1,...,\AnsKeyNum}
	{\item\csname MCAns\romannumeral\i\endcsname}
	\end{enumerate}\end{multicols}
}

\newcommand{\MCVer}[2]{
	\ifnumequal{\MCVerNum}{1}{#1}{#2}
}

\newcommand{\MCContent}{}

\newcommand{\MCBuildem}[2]{
	\newcommand{\MCVerNum}{1}

	\MCStart\MCContent\MCKey{\MCVer{#1}{#2}}

	\newpage
	\setcounter{page}{1}
	\renewcommand{\MCVerNum}{2}
	%HERE

	\MCRestart\MCContent\MCKey{\MCVer{#1}{#2}}
}

\newcommand{\MCBuildit}{
	\newcommand{\MCVerNum}{1}
	\MCStart\MCContent\MCKey{\MCVer{}{}}
}

\iffalse
\ExplSyntaxOn
\newcommand{\fpabs}[1]{\fp_eval:n{abs(#1)}}
\ExplSyntaxOff
\fi

%%%%%%%%%%%%%%%%%%%%%%%%%%%%%%%%%%%%%%%%%%%%%%%%%%%%%%%
%SAMPLE EXCEL RANGE

\newcommand{\excel}{
Use the spreadsheet below to answer the following question.
\begin{center}
\vspace{-11pt}
\begin{tikzpicture}
\draw[gray, thick, fill=gray!25](0,0)--(1,0)--(1,2.25)--(6,2.25)--(6,3)--(0,3)--(0,0);
\draw[gray!60, thick, fill=gray!60](.5,2.35)--(.9,2.35)--(.9,2.75)--(.5,2.35);
\draw[gray, ystep = .75] (0,0) grid (6,3);
\foreach \x in {1, 2, 3, 4, 5}
\foreach \y in {1, 2, 3}
	\node at (.5+\x, 2.625-.75*\y){{\char\numexpr`A-1+\x\relax}\y};
\foreach \x in {1, 2, 3, 4, 5}
	\node at (.5+\x, 2.625){{\char\numexpr`A-1+\x\relax}};
\foreach \y in {1, 2, 3}
	\node at (.5, 2.625-.75*\y){\y};
\end{tikzpicture}
\end{center}
}